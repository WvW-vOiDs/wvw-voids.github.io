\documentclass[aspectratio=169]{beamer}
\usepackage[UTF8,noindent]{ctexcap}  % 中文支持
\usepackage{amsmath,amssymb}         % 数学符号
\usepackage{graphicx}                % 图片插入
\usepackage{listings}                % 代码高亮
\usepackage{algorithm,algorithmic}   % 伪代码
\usepackage{tcolorbox}               % 彩色文本框

% 主题设置
\usetheme{Darmstadt}                 % 灰色主题
\usecolortheme{seagull}              % 灰色配色
\setbeamertemplate{navigation symbols}{}  % 隐藏导航图标

% 标题页信息
\title{Beamer功能全演示}
\subtitle{科研展示最佳实践}
\author{作者名}
\institute{单位名称}
\date{\today}

% 代码块样式
\lstset{
    basicstyle=\ttfamily\footnotesize, % 基本样式
    keywordstyle=\color{blue}, % 关键字样式
    commentstyle=\color{green!50!black}, % 注释样式
    stringstyle=\color{red}, % 字符串样式
    showstringspaces=false, % 不显示字符串中的空格
    breaklines=true, % 自动换行
    frame=single, % 添加边框
    rulecolor=\color{black}, % 边框颜色
    backgroundcolor=\color{gray!10}, % 背景颜色
    numbers=left, % 显示行号
    numberstyle=\tiny\color{gray}, % 行号样式
    stepnumber=1, % 行号步长
    numbersep=5pt % 行号与代码的距离
}

\begin{document}

% 标题页
\begin{frame}[plain]
    \titlepage
\end{frame}

% 目录页
\begin{frame}{目录}
    \tableofcontents[pausesections]  % 分步显示目录
\end{frame}

% 文本设计
\section{文本设计}
\begin{frame}{文本样式}
    \begin{columns}
        \column{0.48\textwidth}
        \begin{tcolorbox}[colback=blue!5!white,colframe=blue!75!black,title=彩色文本框]
            使用tcolorbox宏包实现渐变背景\\
            \textcolor{red}{红字} + \textbf{加粗} + \emph{斜体}
        \end{tcolorbox}

        \column{0.48\textwidth}
        \colorbox{yellow!40}{高亮文本} \\
        \fcolorbox{red}{white}{带边框文本框} \\
        \setlength{\fboxrule}{2pt}
        \fcolorbox{blue}{cyan!20}{自定义边框}
    \end{columns}
\end{frame}

% 代码演示
\section{代码演示}
\begin{frame}[fragile]{Python代码}
\begin{lstlisting}[language=Python]
def quicksort(arr):
    if len(arr) <= 1:
        return arr
    pivot = arr[len(arr) // 2]
    left = [x for x in arr if x < pivot]
    middle = [x for x in arr if x == pivot]
    right = [x for x in arr if x > pivot]
    return quicksort(left) + middle + quicksort(right)
\end{lstlisting}
\end{frame}

\begin{frame}{算法伪代码}
\begin{algorithm}[H]
\caption{快速排序算法}
\begin{algorithmic}[1]
\REQUIRE 无序数组$A$
\ENSURE 有序数组
\IF{$|A| \leq 1$}
    \RETURN $A$
\ENDIF
\STATE 选择基准元素$pivot$
\STATE 划分左/中/右子数组
\RETURN 递归结果拼接
\end{algorithmic}
\end{algorithm}
\end{frame}

% 数学展示
\section{数学展示}
\begin{frame}{定理环境}
    \begin{theorem}[勾股定理]
        直角三角形斜边平方等于两直角边平方和:
        $$ a^2 + b^2 = c^2 $$
    \end{theorem}

    \begin{proof}
        构造边长为$a+b$的正方形,通过面积法证明...
    \end{proof}
\end{frame}

\begin{frame}{图文混排}
    \begin{columns}
        \column{0.6\textwidth}
        \includegraphics[width=\textwidth]{example-image} % 替换为你的图片路径
        \column{0.38\textwidth}
        \begin{itemize}
            \item 图片说明1
            \item 图片说明2
            \item 图片说明3
        \end{itemize}
    \end{columns}
\end{frame}

% 高级功能
\section{高级功能}
\begin{frame}{动画效果}
    \begin{itemize}
        \item 分步显示功能 \pause
        \item \alert<2>{重点强调} \pause
        \item \only<3>{第三步专属内容}
    \end{itemize}

    \onslide<4->{最后呈现的内容}
\end{frame}

\begin{frame}{参考文献}
    \begin{thebibliography}{9}
        \bibitem{ref1} 作者. 标题[J]. 期刊, 年份
        \bibitem{ref2} 作者. 书名[M]. 出版社, 年份
    \end{thebibliography}
\end{frame}

% 结束页
\begin{frame}{结束}
    \centering
    \Huge 谢谢!
\end{frame}

\end{document}